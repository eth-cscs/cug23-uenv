Each software stack is a compressed squashfs image of a directory containing the software, Spack configuration, modules and meta-data.
The squashfs artifacts associated with a programming environment can be versioned, and mounted at the same location using command line utilities or SLURM plugins.
This significantly simplifies testing and deployment of stacks in CI pipelines compared to the current approach of building software stacks on a shared file system with subdirectories for different clusters, where providing multiple versions of a software stack requires creating a directory with a different name.

Other benefits of using a squashfs for images include:
\begin{itemize}
    \item Squashfs images are cached in memory so time to configure and build software is significantly faster than if the tools are on a shared filesystem;
    \item Multiple users on the same node can load different environments on the same mount point.
\end{itemize}

We developed open-source SLURM plugin and command line utilities for mounting squashfs images and configuring the environment, whose implementation will be described in the paper.


