This section presents benchmarks and tests that demonstrate 

%%%%%%%%%%%%%%%%%%%%%%%%%%%%%%%%%%%%%%%%%%%%%%%%%%%%%%%
\subsection{Deployment productivity}
%%%%%%%%%%%%%%%%%%%%%%%%%%%%%%%%%%%%%%%%%%%%%%%%%%%%%%%

This benchmark tests the impact of optimisations to reduce build time of spack stacks described in~\sect{sec:faster-builds}, namely building in memory and caching previously-built packages.

For this demonstration, we built a software stack that has all of the dependencies required to develop Arbor~\cite{paper:arbor2019,software:arbor}, a Neuroscience application written in C++ and Python.
Arbor has a extensive list of dependencies, including C++ libraries, Python and Python packages.

We time the time taken to run make on a clean build, which includes the time taken to bootstrap Spack, concretise and build all of the packages and generate the compressed SquashFS image in four different scenarios:
\begin{itemize}
    \item \textbf{scratch}: Build on an HPE Cray ClusterStor E1000 Scratch file system.
    \item \textbf{memory}: Build in \lst{/dev/shm}, i.e. \emph{in memory}.
    \item \textbf{cache}: Build in \lst{/dev/shm} using a Spack build cache that has all of the packages 
    \item \textbf{partial}: Build in \lst{/dev/shm} using a Spack build cache where the version of Python in the recipe is changed to a version that is not in the cache.
\end{itemize}
Scenario 2 quantifies the effect of building in memory, and scenarios 3 and 4 illustrate the additional benefits of using build caches.

\begin{figure}[htp!]
    \begin{center}
        \begin{tikzpicture}[scale=1]
    \begin{axis} [
        ymin = 0, ymax=3000,
        ybar, bar width = 18pt,
        %symbolic x coords={1,2,3,4},
        grid=major,
        xtick = {1,2,3,4},
        xticklabels = {scratch, memory, cache, partial},
        xticklabel style={text height=1ex},
        ylabel={Build time (s)},
        nodes near coords,
        nodes near coords style={fill=white},
    ]
        \addplot[fill=orange!30] table[x=id, y=time-s] {./data/stack-build.tbl};
        \draw[line width=2pt, ->] (axis cs:2,2714) -- (axis cs:2,1866);
        \draw[line width=2pt, ->] (axis cs:3,1566) -- (axis cs:3,445);
        \draw[line width=2pt, ->] (axis cs:4,1566) -- (axis cs:4,787);
        \node[fill=white] () at (axis cs:2,2300) {\bfseries 1.7x};
        \node[fill=white] () at (axis cs:3,1000) {\bfseries 10.8x};
        \node[fill=white] () at (axis cs:4,1200) {\bfseries 3.2x};

        %\draw (s1_t.center) -- (s1_b.center);
    \end{axis}
\end{tikzpicture}


    \end{center}
    \caption{The effect of building in memory and using Spack build caches on the time to build a complete Spack stack.}
    \label{fig:image-build}
\end{figure}

\fig{fig:image-build} shows that building the image on Scratch takes 45 minutes, which is reduced to 26 minutes when building in memory -- a significant 1.7$\times$ reduction in build time.
Less than 3 minutes are required when all packages are available in a build cache, and less than 10 minutes to build the full stack when the version of Python in the recipe changed, which required rebuilding over 30 packages, including Python, py-numpy and py-mpi4py, which are non-trivial to build.

The \emph{partial} reflects the most common scenario, because the typical CI/CD and image development process requires rebuilding an image with small changes, so that only some of the packages need to be rebuilt between runs.
Furthermore, the bootstrap and compiler toolchains are typically identical between different images -- e.g. once GCC 11.3.0 has been built for one stack, it can be reused without change in another.


%%%%%%%%%%%%%%%%%%%%%%%%%%%%%%%%%%%%%%%%%%%%%%%%%%%%%%%
\subsection{Developer productivity}
%%%%%%%%%%%%%%%%%%%%%%%%%%%%%%%%%%%%%%%%%%%%%%%%%%%%%%%

Squashfs deployment reduces configuration and compilation times compared to serving the same stack on a shared filesystem;

Benchmark configuration time and build time with the same stack stored in different locations:
\begin{itemize}
    \item mounted with squashfs-run
    \item in memory, i.e. \lstinline|/dev/shm|
    \item on scratch.
\end{itemize}

For the following:
\begin{itemize}
    \item simple hello world
    \item medium complexity application (Arbor)
    \item complex application, e.g. DLA-F
\end{itemize}

%%%%%%%%%%%%%%%%%%%%%%%%%%%%%%%%%%%%%%%%%%%%%%%%%%%%%%%
\subsection{Overheads}
%%%%%%%%%%%%%%%%%%%%%%%%%%%%%%%%%%%%%%%%%%%%%%%%%%%%%%%

\assign{Ben}

Quantify the memory overheads and affect on job startup time of mounting squashfs images on compute nodes at the start of SLURM jobs;

%%%%%%%%%%%%%%%%%%%%%%%%%%%%%%%%%%%%%%%%%%%%%%%%%%%%%%%
\subsection{Benchmarks and Applications}
%%%%%%%%%%%%%%%%%%%%%%%%%%%%%%%%%%%%%%%%%%%%%%%%%%%%%%%

\assign{JG (SPH-EXA), Antonk}

Micro-benchmarks and application benchmarks that demonstrate equivalent performance to applications compiled with CPE on the same system.

\noindent\textbf{MicroBenchmark: OSU}

\todo{Ben and Theo}

\noindent\textbf{Application Benchmark: SPH-EXA}

The SPH-EXA\footnote{\url{https://github.com/unibas-dmi-hpc/SPH-EXA}} project is a multidisciplinary effort that looks to scale the Smoothed Particle Hydrodynamics (SPH) method to enable exascale hydrodynamics simulations for the fields of Cosmology and Astrophysics. 
Figure \ref{fig:gpu-A100-best} shows the performance obtained for the MPI+OpenMP+CUDA and MPI+OpenMP+HIP codes executing the Sedov--Taylor\footnote{\url{https://doi.org/10.48550/arxiv.2202.02840}} blast wave explosion test case with $400^3$ particles per gpu and $40$ time-steps.
The results show that the squashfs-based executables delivers competitive performance with that of the cpe based CPU executables on both NVIDIA A100 and AMD MI200 gpus.

\begin{figure*}[htp!]
    \begin{center}
        \begin{tikzpicture}[scale=1]
    \begin{axis} [
        height=6cm, width=8.5cm,
        ymin = 0, ymax = 22,
        ybar, bar width = 10pt,
        symbolic x coords={1,2,3,4},
        xtick = data, grid=major,
        xlabel=Nodes,
        title=\large \bf A100 GPU,
        legend style = {at={(0.95,0.05)}, anchor=south east},
        ylabel={Iterations per minute},
    ]
        \addplot[fill=green!60] table[x=nodes, y=cpe_its_per_min] {./data/sphexa/gpu/results-a100.tbl};
        %\addplot[fill=green!60] table[x=cn, y=it_throughput_per_min] {./data/sphexa/gpu/run-report-A100-64M-CPE-cpe2212.tbl};
        \addlegendentry{cpe/22.12}

        \addplot[fill=orange!60] table[x=nodes, y=uenv_its_per_min] {./data/sphexa/gpu/results-a100.tbl};
        %\addplot[fill=orange!60] table[x=cn, y=it_throughput_per_min] {./data/sphexa/gpu/run-report-A100-64M-SQFS-cpe2212.tbl};
        \addlegendentry{uenv/22.12}
    \end{axis}
\end{tikzpicture}

        \hfill
        \begin{tikzpicture}[scale=1]
    \begin{axis} [
        ymin = 0, ymax = 6.5,
        ybar, bar width = 8pt,
        %x = 0.4cm, % space between bars
        symbolic x coords={1,2,3,4,5,6,7},
        xtick = data, grid=major, % legend pos=outer north east,
        xlabel=AMD MI250x Nodes,
        ylabel={Iterations per minute},
    ]
        % \addplot[fill=green!60] table[x=cn, y=it_throughput_per_min] {./data/sphexa/gpu/run-report-MI200-64M-CPE-cpe2212.tbl};
        \addplot[fill=green!60] table[x=nodes, y=cpe_its_per_min] {./data/sphexa/gpu/results-mi250.tbl};
        \addlegendentry{cpe/22.12}

        % \addplot[fill=orange!60] table[x=cn, y=it_throughput_per_min] {./data/sphexa/gpu/run-report-MI200-64M-SQFS-cpe2212.tbl};
        \addplot[fill=orange!60] table[x=nodes, y=uenv_its_per_min] {./data/sphexa/gpu/results-mi250.tbl};
        \addlegendentry{uenv/22.12}
    \end{axis}
\end{tikzpicture}

        \newline
        \textbf{(a)}
        \hspace{7cm}
        \textbf{(b)}
    \end{center}

    \caption{Weak scaling results on A100 and Mi250x GPU nodes for SPH-EXA (higher is better).}
    \label{fig:shp-weak}
\end{figure*}

%%%%%%%%%%%%%%%%%%%%%%%%%%%%%%%%%%%%%%%%%%%%%%%%%%%%%%%
\subsection{Tools}
%%%%%%%%%%%%%%%%%%%%%%%%%%%%%%%%%%%%%%%%%%%%%%%%%%%%%%%

\assign{JG}

\begin{itemize}
    \item demonstrate DDT
    \item demonstrate a profiler?
\end{itemize}

\subsubsection{Performance tools}

\begin{table}[htp!]
    \centering
    \pgfplotstabletypeset[
        precision=0,
        every head row/.style={before row=\toprule,after row=\midrule},
        col sep=space, columns={num-tasks, H2D-MB-max, D2H-MB-max, D2D-MB-max},
        columns/num-tasks/.style={column name=A100},
        columns/H2D-MB-max/.style={column name=H2D},
        columns/D2H-MB-max/.style={column name=D2H},
        columns/D2D-MB-max/.style={column name=D2D},
    ]
        {./data/sphexa/tbl/run-report-A100-64M-SQFS-cpe2302-NSYS.tbl}
    \caption{hello world}
    \label{table:nsys-A100}
\end{table}

Table \ref{table:nsys-A100} shows the amount of CUDA memory copies (in MB) reported by NVIDIA performance tool (Nsight Systems) for the Sedov--Taylor test case.
The transfer sizes (in MB) for Host to Device (H2D), Device to Host (D2H) and Device to Device (D2D) for simulations with uenv and without uenv (cpe only) are equal, demonstrating that the performance tool can be used in both scenarios.

%%%%%%%%%%%%%%%%%%%%%%%%%%%%%%%%%%%%%%%%%%%%%%%%%%%%%%%
\section{Future work}
%%%%%%%%%%%%%%%%%%%%%%%%%%%%%%%%%%%%%%%%%%%%%%%%%%%%%%%

We will discuss collaboration with HPE to provide Cray-MPICH and other HPE software packages via spack stacks, and how we plan to deliver software for the large Grace-Hopper scale out in the second half of 2023 at CSCS.

%%% Local Variables:
%%% TeX-master: "paper"
%%% End:
