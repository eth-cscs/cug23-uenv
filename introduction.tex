The CSCS User Lab provides a scientific computing services to a diverse domains and use cases, each requiring different programming languages, libraries and tools.
Currently, the programming environment is based on the Cray Programming Environment (CPE), which provides a very wide range of tools including compilers, scientific libraries, communication libraries, and tools.
The CPE continues to evolve and expand in response to changing user requirements and node-types.
CSCS provides additional software built using the CPE, all deployed in a common module environment.

There are some drawbacks to this approach of providing a large, one-size-fits-all environment, that make it difficult to support as the number of use cases and hardware architectures provided on a HPC system scales out:
\begin{itemize}
    \item The CPE is complicated, so that it is not practical to test many of its configurations;
    \item No use-case or domain will use more than a small subset of the features provided by the CPE;
    \item The CPE released on a quarterly cycle, so the lead time between identifying an issue and a fix available and tested on site can be expected to be in the order of 3-6 months;
\end{itemize}
By striking a balance between long term stability and providing the latest versions and tools, it does not satisfy use cases that require either.

CSCS is deploying isolated clusters (vClusters) on the HPE Cray EX system called Alps, to provide services to a wider range of user domains, each with their own software, security, reliability and scaling requirements.
The vClusters can be customized for the target use cases, as an alternative to providing one large system with a "one size fits all" programming environment, storage and job scheduler.
This gives CSCS the opportunity to customize the programming environment for vClusters; specifically by providing the smallest possible set of compilers, libraries and tools optimized for the cluster's requirements, node architecture and the Slingshot 11 interconnect.

In this paper we will describe a method for defining, building and deploying alternative programming environments alongside the CPE on HPE Cray EX systems.
We provide compact, testable, optimized software environments based on a descriptive recipe that can be updated independently of the CPE release cycle.
The environments are deployed as squashfs images, that users can mount using a SLURM plugin and command line utilities.
